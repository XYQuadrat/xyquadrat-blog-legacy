\documentclass[aspectratio=169]{beamer}

\usetheme[titleformat=smallcaps, block=fill]{metropolis}

\usepackage[ngerman]{babel}
\usepackage{booktabs}

\usepackage{amsmath}
\usepackage[mathrm=sym]{unicode-math}

\usepackage{tikz}
\usepackage{pgfplots}
\usepackage{pgfplotsthemetol}
\pgfplotsset{compat=newest}

\setmathfont{Fira Math}
\usepackage{hyperref}
\hypersetup{colorlinks=true,linkcolor=,urlcolor=blue}

\title{Algorithmen \& Datenstrukturen \\ Woche 1}
\date{27. September 2021}
\author{Julian Steinmann}
\institute{ETH Zürich}

\begin{document}
  \maketitle
  \section{Organisatorisches}
  \begin{frame}{Kontakt}
    \begin{itemize}
      \item In Übungsstunde
      \item Mail: \href{mailto:jsteinmann@student.ethz.ch}{jsteinmann@student.ethz.ch}
      \item Discord: \url{@Julian / xyquadrat [A&D]}
    \end{itemize}
    Ausserdem: Alle Slides und andere Materialien auf \href{https://xyquadrat.ch/and}{https://xyquadrat.ch/and} verfügbar.
  \end{frame}
  \begin{frame}{Bonusaufgaben}
    \begin{itemize}
      \item Wöchentlich Sheet mit Bonusaufgaben und optionalen Challenge-Aufgaben (markiert mit *)
      \item In zufälligen Zweiergruppen lösen (wechseln alle 3 Wochen)
      \item Verteilung der Bonuspunkte
      \begin{itemize}
        \item 3 für Bonusaufgaben
        \item 1 für Peer Grading
        \item 4 für Programmieraufgaben (erst später)
      \end{itemize}
      \item 80\% aller Bonuspunkte \(\rightarrow\) +0.25 in Prüfung
      \item Abgabe bis zu Beginn der Stunde, am Besten auf Papier \textit{und} digital
    \end{itemize}
  \end{frame}
  \begin{frame}{Peer Grading}
    Grundsätzlich: von 11:15 - 12:00. Meist früher fertig. \\
    Aber: Abgabefrist ist Montag, 23:59 falls nötig.
  \end{frame}
  \begin{frame}{Schwierigkeit von A\&D}
    \begin{center}
    \begin{tabular}{lrr}
      \toprule
      & HS19 & HS20 \\
      \midrule
      Diskrete Mathematik & 3.88 & 3.72 \\
      Einführung in die Programmierung & 4.31 & 4.24 \\
      Lineare Algebra & 4.27 & 4.09 \\
      \textbf{Algorithmen und Datenstrukturen} & 4.16 & 4.19 \\
      \bottomrule
    \end{tabular}
    \end{center}
  \end{frame}
  \section{Induktion}
  \begin{frame}{Was ist Induktion?}
    Induktion ist eine Art, Aussagen zu beweisen, beispielsweise \[\sum_{k=1}^n k = \frac{n(n+1)}{2}\] Wir benötigen zwei Dinge, um Induktion anzuwenden:
    \begin{enumerate}
      \item Die Aussage muss für einen Basisfall stimmen. (Oft \(n=0\) oder \(n=1\))
      \item Wenn die Aussage für einen Fall \(n\) stimmt, dann muss sie auch für den nächsten Fall stimmen (Oft \(n+1\) oder \(2\times n\))
    \end{enumerate}
  \end{frame}
  \begin{frame}[standout]
    Beispiel für Induktionsbeweis
  \end{frame}
  \section{Asymptotisches Wachstum}
  \begin{frame}{Kleine Eingaben}
    \begin{center}
      \begin{tikzpicture}
        \begin{axis}[xmin=0,xmax=100,ymin=0,ymax=1000,samples=50,grid=major,xlabel={Eingabegrösse},ylabel={Operationen},mlineplot]
        \addplot[TolDarkBlue, thick, domain=0:100]{0.05*x*x};
        \addplot[TolLightRed, thick, domain=0:100]{x*log2(x)};
        
        \legend{
          \(0.5x^2\),
          \(x \log x\)
        }
        \end{axis}
      \end{tikzpicture}
    \end{center}
  \end{frame}
  \begin{frame}{Grosse Eingaben}
    \begin{center}
      \begin{tikzpicture}
        \begin{axis}[xmin=0,xmax=1000,ymin=0,ymax=100000,samples=50,grid=major,xlabel={Eingabegrösse},ylabel={Operationen},mlineplot]
        \addplot[TolDarkBlue, thick, domain=0:1000]{0.05*x*x};
        \addplot[TolLightRed, thick, domain=0:1000]{x*log2(x)};
        
        \legend{
          \(0.5x^2\),
          \(x \log x\)
        }
        \end{axis}
      \end{tikzpicture}
    \end{center}
  \end{frame}
  \begin{frame}[standout]
    Beispiel für Asymptotisches Wachstum
  \end{frame}
\end{document}
